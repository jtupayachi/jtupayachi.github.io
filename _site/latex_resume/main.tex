\documentclass[10pt, letterpaper]{article}

% Packages:
\usepackage[
    ignoreheadfoot, % set margins without considering header and footer
    top=2 cm, % seperation between body and page edge from the top
    bottom=2 cm, % seperation between body and page edge from the bottom
    left=2 cm, % seperation between body and page edge from the left
    right=2 cm, % seperation between body and page edge from the right
    footskip=1.0 cm, % seperation between body and footer
    % showframe % for debugging 
]{geometry} % for adjusting page geometry
\usepackage{titlesec} % for customizing section titles
\usepackage{tabularx} % for making tables with fixed width columns
\usepackage{array} % tabularx requires this
\usepackage[dvipsnames]{xcolor} % for coloring text
\definecolor{primaryColor}{RGB}{0, 0, 0} % define primary color
\usepackage{enumitem} % for customizing lists
\usepackage{fontawesome5} % for using icons
\usepackage{amsmath} % for math
\usepackage{multicol}
\usepackage[
    pdftitle={John Doe's CV},
    pdfauthor={John Doe},
    pdfcreator={LaTeX with RenderCV},
    colorlinks=true,
    urlcolor=primaryColor
]{hyperref} % for links, metadata and bookmarks
\usepackage[pscoord]{eso-pic} % for floating text on the page
\usepackage{calc} % for calculating lengths
\usepackage{bookmark} % for bookmarks
\usepackage{lastpage} % for getting the total number of pages
\usepackage{changepage} % for one column entries (adjustwidth environment)
\usepackage{paracol} % for two and three column entries
\usepackage{ifthen} % for conditional statements
\usepackage{needspace} % for avoiding page brake right after the section title
\usepackage{iftex} % check if engine is pdflatex, xetex or luatex

% Ensure that generate pdf is machine readable/ATS parsable:
\ifPDFTeX
    \input{glyphtounicode}
    \pdfgentounicode=1
    \usepackage[T1]{fontenc}
    \usepackage[utf8]{inputenc}
    \usepackage{lmodern}
\fi

\usepackage{charter}

% Some settings:
\raggedright
\AtBeginEnvironment{adjustwidth}{\partopsep0pt} % remove space before adjustwidth environment
\pagestyle{empty} % no header or footer
\setcounter{secnumdepth}{0} % no section numbering
\setlength{\parindent}{0pt} % no indentation
\setlength{\topskip}{0pt} % no top skip
\setlength{\columnsep}{0.15cm} % set column seperation
\pagenumbering{gobble} % no page numbering

\titleformat{\section}{\needspace{4\baselineskip}\bfseries\large}{}{0pt}{}[\vspace{1pt}\titlerule]

\titlespacing{\section}{
    % left space:
    -1pt
}{
    % top space:
    0.3 cm
}{
    % bottom space:
    0.2 cm
} % section title spacing

\renewcommand\labelitemi{$\vcenter{\hbox{\small$\bullet$}}$} % custom bullet points
\newenvironment{highlights}{
    \begin{itemize}[
        topsep=0.10 cm,
        parsep=0.10 cm,
        partopsep=0pt,
        itemsep=0pt,
        leftmargin=0 cm + 10pt
    ]
}{
    \end{itemize}
} % new environment for highlights


\newenvironment{highlightsforbulletentries}{
    \begin{itemize}[
        topsep=0.10 cm,
        parsep=0.10 cm,
        partopsep=0pt,
        itemsep=0pt,
        leftmargin=10pt
    ]
}{
    \end{itemize}
} % new environment for highlights for bullet entries

\newenvironment{onecolentry}{
    \begin{adjustwidth}{
        0 cm + 0.00001 cm
    }{
        0 cm + 0.00001 cm
    }
}{
    \end{adjustwidth}
} % new environment for one column entries

\newenvironment{twocolentry}[2][]{
    \onecolentry
    \def\secondColumn{#2}
    \setcolumnwidth{\fill, 4.5 cm}
    \begin{paracol}{2}
}{
    \switchcolumn \raggedleft \secondColumn
    \end{paracol}
    \endonecolentry
} % new environment for two column entries

\newenvironment{threecolentry}[3][]{
    \onecolentry
    \def\thirdColumn{#3}
    \setcolumnwidth{, \fill, 4.5 cm}
    \begin{paracol}{3}
    {\raggedright #2} \switchcolumn
}{
    \switchcolumn \raggedleft \thirdColumn
    \end{paracol}
    \endonecolentry
} % new environment for three column entries

\newenvironment{header}{
    \setlength{\topsep}{0pt}\par\kern\topsep\centering\linespread{1.5}
}{
    \par\kern\topsep
} % new environment for the header

\newcommand{\placelastupdatedtext}{% \placetextbox{<horizontal pos>}{<vertical pos>}{<stuff>}
  \AddToShipoutPictureFG*{% Add <stuff> to current page foreground
    \put(
        \LenToUnit{\paperwidth-2 cm-0 cm+0.05cm},
        \LenToUnit{\paperheight-1.0 cm}
    ){\vtop{{\null}\makebox[0pt][c]{
        \small\color{gray}\textit{Last updated in September 2024}\hspace{\widthof{Last updated in September 2024}}
    }}}%
  }%
}%

% save the original href command in a new command:
\let\hrefWithoutArrow\href

% new command for external links:


\begin{document}
    \newcommand{\AND}{\unskip
        \cleaders\copy\ANDbox\hskip\wd\ANDbox
        \ignorespaces
    }
    \newsavebox\ANDbox
    \sbox\ANDbox{$|$}

    \begin{header}
        \fontsize{25 pt}{25 pt}\selectfont Jose Tupayachi

        \vspace{5 pt}

        \normalsize
        \mbox{Knoxville, TN}%
        \kern 5.0 pt%
        \AND%
        \kern 5.0 pt%
        \mbox{\hrefWithoutArrow{mailto:jtupayac@vols.utk.edu}{jtupayac@vols.utk.edu}}%
        \kern 5.0 pt%
        \AND%
        \kern 5.0 pt%
        \mbox{\hrefWithoutArrow{tel:+1 661 365 5289}{661 365 5289}}%
        \kern 5.0 pt%
        \AND%
        \kern 5.0 pt%
        \mbox{\hrefWithoutArrow{https://jtupayachi.github.io/}{jtupayachi.github.io}}%
        % \kern 5.0 pt%
        % \AND%
        % \kern 5.0 pt%
        % \mbox{\hrefWithoutArrow{https://linkedin.com/in/https://www.linkedin.com/in/jose-tupayachi-582339147}{linkedin.com/in/jose-tupayachi-582339147}}%
        % \kern 5.0 pt%
        % \AND%
        % \kern 5.0 pt%
        % \mbox{\hrefWithoutArrow{https://github.com/jtupayachi}{github.com/jtupayachi}}%
    \end{header}

    \vspace{5 pt }


% Introduction:
% \section*{Introduction}
Experienced Python developer and Large Language Model (LLM) expert with a strong track record in designing, implementing, and deploying advanced machine learning systems. Skilled in leveraging Retrieval-Augmented Generation (RAG), vector databases, and API-driven solutions to enhance real-time NLP applications. Proven collaborator with interdisciplinary teams, focused on delivering impactful, data-driven products.

% \section*{Areas of Research}

        
        % \begin{onecolentry}
        %     \href{https://rendercv.com}{RenderCV} is a LaTeX-based CV/resume version-control and maintenance app. It allows you to create a high-quality CV or resume as a PDF file from a YAML file, with \textbf{Markdown syntax support} and \textbf{complete control over the LaTeX code}.
        % \end{onecolentry}

        % \vspace{0.2 cm}

        % \begin{onecolentry}
        %     The boilerplate content was inspired by \href{https://github.com/dnl-blkv/mcdowell-cv}{Gayle McDowell}.
        % \end{onecolentry}


    
    % \section{Quick Guide}

    % \begin{onecolentry}
    %     \begin{highlightsforbulletentries}


    %     \item Each section title is arbitrary and each section contains a list of entries.

    %     \item There are 7 unique entry types: \textit{BulletEntry}, \textit{TextEntry}, \textit{EducationEntry},  \textit{NormalEntry}, \textit{PublicationEntry}, and \textit{OneLineEntry}. %\textit{ExperienceEntry},

    %     \item Select a section title, pick an entry type, and start writing your section!

    %     \item \href{https://docs.rendercv.com/user_guide/}{Here}, you can find a comprehensive user guide for RenderCV.


    %     \end{highlightsforbulletentries}
    % \end{onecolentry}


% Education:
\section*{Education}
\begin{twocolentry}{Aug 2024 - Present}
    \small \textbf{University of Tennessee, Knoxville} \textbar PhD Candidate in Industrial and Systems Engineering \\
    Advisor: Dr. Xueping Li, Co-advised: Dr. Haowen Xu 
\end{twocolentry}

\begin{twocolentry}{Aug 2022 - Jul 2024}
    \small \textbf{University of Tennessee, Knoxville} \textbar MS in Industrial and Systems Engineering\\
    GPA: 3.9
\end{twocolentry}

\begin{twocolentry}{Mar 2014 – Jan 2020}
    \small \textbf{Pontifical Catholic University of Peru} \textbar BS in Industrial Engineering
\end{twocolentry}


% \begin{highlights}
%     \item \textbf{Coursework:} Systems Engineering, Quality Control
% \end{highlights}





    % \section{Education}



        
    %     \begin{twocolentry}{
    %         Sept 2000 – May 2005
    %     }
    %         \textbf{University of Pennsylvania}, BS in Computer Science\end{twocolentry}

    %     \vspace{0.10 cm}
    %     \begin{onecolentry}
    %         \begin{highlights}
    %             \item GPA: 3.9/4.0 (\href{https://example.com}{a link to somewhere})
    %             \item \textbf{Coursework:} Computer Architecture, Comparison of Learning Algorithms, Computational Theory
    %         \end{highlights}
    %     \end{onecolentry}



    
    % \section{Experience}



        
    %     \begin{twocolentry}{
    %         June 2005 – Aug 2007
    %     }
    %         \textbf{Software Engineer}, Apple -- Cupertino, CA\end{twocolentry}

    %     \vspace{0.10 cm}
    %     \begin{onecolentry}
    %         \begin{highlights}
    %             \item Reduced time to render user buddy lists by 75\% by implementing a prediction algorithm
    %             \item Integrated iChat with Spotlight Search by creating a tool to extract metadata from saved chat transcripts and provide metadata to a system-wide search database
    %             \item Redesigned chat file format and implemented backward compatibility for search
    %         \end{highlights}
    %     \end{onecolentry}


    %     \vspace{0.2 cm}

    %     \begin{twocolentry}{
    %         June 2003 – Aug 2003
    %     }
    %         \textbf{Software Engineer Intern}, Microsoft -- Redmond, WA\end{twocolentry}

    %     \vspace{0.10 cm}
    %     \begin{onecolentry}
    %         \begin{highlights}
    %             \item Designed a UI for the VS open file switcher (Ctrl-Tab) and extended it to tool windows
    %             \item Created a service to provide gradient across VS and VS add-ins, optimizing its performance via caching
    %             \item Built an app to compute the similarity of all methods in a codebase, reducing the time from $\mathcal{O}(n^2)$ to $\mathcal{O}(n \log n)$
    %             \item Created a test case generation tool that creates random XML docs from XML Schema
    %             \item Automated the extraction and processing of large datasets from legacy systems using SQL and Perl scripts
    %         \end{highlights}
    %     \end{onecolentry}


    \section*{Publications}

    \begin{itemize}[left=0pt] % Align items to the left
        \item \textbf{Drone-aided delivery methods, challenges, and the future: A methodological review} \\
        X Li, J Tupayachi, A Sharmin, M Martinez Ferguson \\
        \textit{Drones 7 (3), 191} (2023) \\
        % DOI: \url{https://doi.org/10.xxxx/xxxxx}
    
        \item \textbf{Towards next-generation urban decision support systems through AI-powered construction of scientific ontology using large language models—A case in optimizing intermodal freight} \\
        J Tupayachi, H Xu, OA Omitaomu, MC Camur, A Sharmin, X Li \\
        \textit{Smart Cities 7 (5), 2392-2421} (2024) \\
        % DOI: \url{https://doi.org/10.xxxx/xxxxx}
    
        \item \textbf{Automating Bibliometric Analysis with Sentence Transformers and Retrieval-Augmented Generation (RAG): A Pilot Study in Semantic and Contextual Search for Customized Literature} \\
        H Xu, X Li, J Tupayachi, JJ Lian, OA Omitaomu \\
        \textit{Proceedings of the 2nd ACM SIGSPATIAL International Workshop on Advances in …} (2024) \\
        % DOI: \url{https://doi.org/10.xxxx/xxxxx}
    
        \item \textbf{Better Efficiency on Non-performing Loans Debt Recovery and Portfolio Valuation Using Machine Learning Techniques} \\
        J Tupayachi, L Silva \\
        \textit{Production and Operations Management: POMS Lima, Peru, December 2-4, 2021} (2022) \\
        % DOI: \url{https://doi.org/10.xxxx/xxxxx}
    
    
        \item \textbf{A Simulation-Based Real-Time Deep Reinforcement Learning Approach for Fighting Wildfires} \\
        J Tupayachi, MM Ferguson, X Li \\
        \textit{2024 Annual Modeling and Simulation Conference (ANNSIM), 1-12} \\
        % DOI: \url{https://doi.org/10.23919/ANNSIM61499.2024.10732085}
    \end{itemize}
    

    \section*{Pre-Prints:}

    \begin{itemize}[left=0pt] 

        \item \textbf{Empowering Cognitive Digital Twins with Generative Foundation Models: Developing a Low-Carbon Integrated Freight Transportation System} \\
        X Li, H Xu, J Tupayachi, O Omitaomu, X Wang \\
        \textit{arXiv preprint arXiv:2410.18089} (2024) \\
        % DOI: \url{https://doi.org/10.xxxx/xxxxx}



        \item \textbf{Towards Next-Generation Urban Decision Support Systems through AI-Powered Generation of Scientific Ontology using Large Language Models: A Case in Optimizing Intermodal Freight Transportation} \\
        J Tupayachi, H Xu, O A Omitaomu, M C Camur, A Sharmin, X Li \\
        \textit{arXiv preprint arXiv:2405.19255} (2024)

    \end{itemize}
    



        \section*{Funded Projects Developer}

        \begin{twocolentry}{Nov 2021 – Present} \textbf{RECOIL} \textbar Cognitive Freight Transportation Digital Twin for Resiliency and Emission Control Through Optimizing Intermodal Logistics \end{twocolentry} \begin{highlights} \item Developed and integrated digital twin models for large-scale freight transportation networks, leveraging data from diverse transport modes (road, rail, waterway) to enhance system performance, minimize emissions, and increase resilience. 
            \item Utilized advanced forecasting models to analyze transportation systems' responses to climate change and extreme weather events, enabling data-driven, real-time decision-making such those applicable to EV charging stations. 
            \item Applied Large Language models to improve transportation planning decision making, enabling more accurate tradeoff analysis among cost, time, and emissions. 
            \item Designed a dynamic real-time feedback loop for digital twins, utilizing sensor data and scrapping technology to continuously optimize system performance based on real-world conditions.
             \item Collaborated with academic researchers, the Oak Ridge National Laboratory, and industry partners to develop scalable solutions for technology transfer, focusing on reducing emissions and enhancing resilience in freight and supply chains. \\
            %  \item Led a multi-institutional research initiative funded by the U.S. Department of Energy's ARPA-E program, advancing efforts to decarbonize transportation and tackle critical issues in the freight industry. 
            \end{highlights} \vspace{0.3cm} \noindent \textbf{Funding Agency:} U.S. Department of Energy’s Advanced Research Projects Agency-Energy (ARPA-E)\\  \textbf{Project Number:} \#DE-AR0001780

        \vspace{0.3cm} \begin{twocolentry}{December 2022 – Present} \textbf{SmartShots} \textbar Cross-Platform Application to Improve Childhood Vaccination Rates in Tennessee \end{twocolentry} \begin{highlights} \item Enhanced vaccination tracking with real-time data updates, notifications, and integrated alerts for users and healthcare providers. \item Developed a scalable backend system using Dart and Flutter, ensuring smooth functionality across platforms and enhancing the user experience. \item Integrated community health data to offer users real-time access to nearby vaccination providers and appointment availability. \item Collaborated with the Tennessee Department of Health and local health agencies to align the app with state public health objectives and needs. \item Led user testing and incorporated feedback to improve the app’s inclusivity, addressing issues such as misinformation and disparities in healthcare access. \end{highlights} \vspace{0.3cm} \noindent \textbf{Funding Agency:} Tennessee Department of Health
        
        \vspace{0.3cm} \begin{twocolentry}{Jul 2024 - Present} \textbf{Active Caregiver’s Toolkit (ACTAPP)} \textbar Mobile Application to Promote Physical Activity Among Rural Appalachian Caregivers at Risk for Cardiovascular Disease (CVD) \end{twocolentry} \begin{highlights} \item Development of the ACTAPP as a digital solution for rural Appalachian caregivers, aiming to reduce cardiovascular disease risks through targeted physical activity interventions. 
            % \item Conducted usability testing and a pilot study to assess the app’s effectiveness in promoting healthier lifestyles and reducing CVD risk among caregivers in rural communities. 
        \end{highlights} \vspace{0.3cm} \noindent \textbf{Funding Agency:} Hillman Emergent Innovation (HEI)
% \noindent \textbf{Project Overview:} 
% The project aims to digitize the successful HeartHealth intervention into a mobile app (ACTAPP) to support rural Appalachian caregivers in improving their physical activity, which is crucial for cardiovascular disease (CVD) prevention. The app will be tested and refined based on caregiver feedback to ensure usability, accessibility, and effectiveness in promoting physical activity.

% \vspace{0.3cm}

% \noindent \textbf{Background:} 
% Caregivers in rural Appalachia face higher CVD risks due to limited access to healthcare and higher stress levels. This project will address these issues by providing an on-demand, easily accessible mobile application to promote physical activity among caregivers.

% \vspace{0.3cm}

% \noindent \textbf{Study Approach:}
% \begin{itemize}
%     \item \textbf{Aim 1:} Develop the digitalized Physical Activity Module (PAM) for the app using Design Thinking principles.
%     \item \textbf{Aim 2:} Conduct usability studies with rural caregivers to assess app effectiveness, usability, and satisfaction.
%     \item \textbf{Aim 3:} Pilot test the app to evaluate its impact on key CVD risk factors, including blood pressure and stress.
% \end{itemize}

% \vspace{0.3cm}

% \noindent \textbf{Expected Outcome:} 
% A mobile app that provides an effective, evidence-based intervention to improve physical activity and reduce CVD risk for rural caregivers, with the potential for wider adoption through app stores.




        \section*{Awards \& Scholarships}

        \noindent \textbf{IISE Data Analytics \& Information Systems (DAIS) Student Mobile App Competition} \\
        \textit{2024 Winners - SmartShots Project, Montreal} \\
        
        % https://www.iise.org/Details.aspx?id=33697
        
        \vspace{0.5em} % Adds space between sections
        
        \noindent \textbf{Graduate Fellowships and Awards} \\
        \textit{Holiday Fellowship}: 2022, 2023, 2024 \\
        
        \vspace{0.5em} % Adds space between sections
        
        \noindent \textbf{HIDA Helmholtz Visiting Researcher} \\
        \textit{Year}: 2024 - Awarded but not taken.

        
        % Work Experience:
\section*{Work Experience}

\begin{twocolentry}{Jan 2022 – Aug 2022}
    \textbf{Data Engineer} \textbar Indra – Full-time
\end{twocolentry}
\begin{highlights}
    \item Developed and maintained data pipelines using Python and Shell scripting to streamline big data workflows.
    \item Worked with Apache Spark, Hadoop, and HQL for distributed data processing, querying, and large-scale data migration, including data migration from Oracle and SaaS to PySpark.
    \item Implemented Jenkins-based deployment strategies for automating ETL processes and job scheduling.
    \item Ensured data quality and performance through rigorous data governance practices and optimization techniques.
    \item Managed memory allocation for distributed processing tasks to optimize resource utilization and improve processing efficiency.
\end{highlights}


\begin{twocolentry}{Nov 2020 – Dec 2021}
    \textbf{Data Analyst} \textbar Enel Group – Full-time
\end{twocolentry}
\begin{highlights}
    \item Improved collection systems using a "Payments to Customer" strategy, leveraging unsupervised clustering techniques to enhance efficiency.
    \item Implemented dashboards PowerBi enabling strategic insights at Enel Peru.
    \item Managed SQL and T-SQL databases, ensuring data quality and accurate reporting for Enel's Business Partners in Salesforce.
    \item Developed a desktop application PyQT to streamline invoice collection verification, and automated digital invoice processing.
\end{highlights}

\vspace{0.3cm}






\section*{Community Service}

\textbf{University of Tennessee Graduate Student Senate} \\  
Senator, Industrial and Systems Engineering, 2024–2025 \\  
Represented ISE students, advocated for academic and professional development opportunities, and fostered graduate student engagement across campus.



\section*{Conferences \& Paper Review}
\textbf{Paper Reviewer:}\\
Scenario Decomposition Approach for Mobile Multi-Agent Monitoring under Failure
Submitted to \textit{Transportation Research Part C: Emerging Technologies - November 2024} \\


\textbf{Conference Presenter:}\\
Federated Learning Fault Detection: Towards a Decentralized Machine Learning Framework - IISE 2023\\
Rule-based Automated Cancer Staging from Scanned Pathology Reports - IISE 2024\\
Empowering Simulation Modeling: An Automated Ontology Framework Enhanced by Large Language Models - INFORMS 2024\\
Conversational Geographic Question Answering: LLMs \& Continuous Retrieval-Augmented Generation - SIGSPATIAL 2024\\


% Technologies:
\section*{Technologies}
\begin{multicols}{2}
    \textbf{Languages:} Python, SQL, Bash, Dart, PHP \\
    \textbf{Frameworks:} Django, Flask, TensorFlow, PyTorch, Flutter, Laravel, React \\
    \textbf{ML/NLP Tools:} LlamaIndex, LangChain, HuggingFace Transformers, RAG \\
    \textbf{Databases:} PostgreSQL, MongoDB, Elasticsearch, MySQL \\
    \textbf{DevOps:} Docker, Kubernetes, Git, Jenkins \\
    \textbf{Optimization:} Networkx, Gurobi, cplex
\end{multicols}

    

\end{document}